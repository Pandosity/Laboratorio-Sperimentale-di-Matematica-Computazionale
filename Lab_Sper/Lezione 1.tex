\documentclass{article}
\usepackage[a4paper, top=3cm, left=3cm, bottom=3cm, right=3cm]{geometry}
\usepackage[T1]{fontenc} 
\usepackage[utf8]{inputenc} 
\usepackage[italian]{babel} 
\usepackage{lipsum} 
\usepackage{url} 
\usepackage{lmodern}
\usepackage{graphicx}
\usepackage{psfrag}
\usepackage{amsfonts}
\usepackage{amsthm}
\usepackage{amsmath}
\usepackage{amssymb}
\usepackage{mathrsfs}
\usepackage{tikz-cd}
\usepackage{mathtools}
\usepackage{tikz} 
\usepackage{caption}
\captionsetup{labelformat=empty,textfont=sl}
\usetikzlibrary{decorations.markings}
\usepackage{enumitem}
\usepackage[hidelinks]{hyperref}
\usepackage[numbered,framed]{matlab-prettifier}

\lstset{
  style              = Matlab-editor,
  basicstyle         = \mlttfamily,
  escapechar         = ",
  mlshowsectionrules = true,
}

\title{\textbf{Laboratorio Sperimentale di Matematica Computazionale / / Esercitazione 1}}
\author{Dario Rancati} 


\begin{document}

\maketitle


\section*{Esercizio 1}

La function che risolve il problema è la seguente. Essa divide l'intervallo \texttt{slot} nei nodi
del metodo di Eulero con passo $h$, in ogni nodo calcola la derivata della soluzione, 
 moltiplica questo valore per $h$ e infine lo aggiunge al valore della soluzione.

\begin{lstlisting}[style=Matlab-editor]
function [x,u]=eulero(odefun,slot,y0,h)
L=floor((slot(2)-slot(1))/h);
x=zeros(1,L+1);
x(1)=slot(1);
for i=2:L+1
    x(i)=x(i-1)+h;
end
u=zeros(1,L+1);
u(1)=y0;
for i=2:L+1
    w=feval(odefun, x(i-1),u(i-1));
    u(i)=h*w+u(i-1);
end
end
\end{lstlisting}


\section*{Esercizio 2}

Il corpo del file \textit{function.m} è il seguente:
\begin{lstlisting}[style=Matlab-editor]
function yp = odefun(x,y)
yp = -(2*y+x^2*y^2)/x;
end
\end{lstlisting}

\noindent
Il seguente è lo script che disegna la soluzione esatta in blu e quella approssimata in rosso. Abbiamo inoltre introdotto e plottato la funzione $z$ che misura quanto le due soluzioni si discostano l'una dall'altra (abbiamo riutilizzato le notazioni e la funzione \texttt{eulero} dell'esercizio 1):

\begin{lstlisting}[style=Matlab-editor]
h=0.1;
y0=1;
fun=@(x,y) odefun(x,y);
slot=[1,2];
[x,u]=eulero(fun,slot,y0,h);
y=1./((x.^2).*(log(x)+1));
z=y-u;
plot(x,u,'r',x,y,'b',x,z,'g');
legend('soluzione approssimata', 'soluzione esatta','errore');
\end{lstlisting}

\noindent
Se si esegue lo script, si ottiene la seguente immagine in output:

\begin{figure}[!h]
\centering
\includegraphics[width=16cm]{figura2_1.jpg}
\end{figure}

\noindent
Il grafico mostra chiaramente che l'errore ottenuto con questo metodo è piuttosto consistente: la soluzione esatta e quella approssimata si discostanodi un valore che, a $x=2$, si vede essere ancora essere dell'ordine di grandezza $10^{-1}$.

\end{document}
