\documentclass{article}
\usepackage[a4paper, top=3cm, left=3cm, bottom=3cm, right=3cm]{geometry}
\usepackage[T1]{fontenc} 
\usepackage[utf8]{inputenc} 
\usepackage[italian]{babel} 
\usepackage{lipsum} 
\usepackage{url} 
\usepackage{lmodern}
\usepackage{graphicx}
\usepackage{psfrag}
\usepackage{amsfonts}
\usepackage{amsthm}
\usepackage{amsmath}
\usepackage{amssymb}
\usepackage{mathrsfs}
\usepackage{tikz-cd}
\usepackage{mathtools}
\usepackage{tikz} 
\usepackage{caption}
\captionsetup{labelformat=empty,textfont=sl}
\usetikzlibrary{decorations.markings}
\usepackage{enumitem}
\usepackage[hidelinks]{hyperref}
\usepackage[numbered,framed]{matlab-prettifier}

\lstset{
  style              = Matlab-editor,
  basicstyle         = \mlttfamily,
  escapechar         = ",
  mlshowsectionrules = true,
}

\title{\textbf{Laboratorio Sperimentale di Matematica Computazionale / / Esercitazione 7}}
\author{Dario Rancati} 


\begin{document}




\section*{Esercizio 1}
La function calcola le soluzioni di $y'=\cos(x^2)$ e di $z'=\sin(x^2),$ sia avanti che indietro nel tempo. Abbiamo scelto $xmax = 150$ in quanto le soluzioni si avvicinano chiaramente ai due punti limite (gli integrali di Fresnel), che la funztion restituisce e che valgono $0.6264;0.6233.$ e il simmetrico rispetto all'origine (l'intera soluzione è infatti simmetrica rispetto ad essa).
\begin{lstlisting}
function l=spiraleeulero
fun=@(x,y) [cos(x^2);sin(x^2)];
revfun=@(x,y) [-cos(x^2);-sin(x^2)];
options=odeset('RelTol',1e-10);
[~,y1]=ode45(fun,[0 150],[0;0],options);
[~,y2]=ode45(revfun,[0 150],[0;0],options);
y=[flip(y2);y1];
plot(y(:,1),y(:,2),'r');
axis equal
axis([-1 1 -1 1])
title('Spirale di Eulero');
l=y1(end,:);
end
\end{lstlisting}

\begin{figure}[!h]
\centering
\includegraphics[width=\textwidth]{figura_7_1.jpg}
\end{figure}

\section*{Esercizio 2}
L'equazione che ho scelto in modo che abbia come soluzione la spirale è: 
\begin{equation}
\begin{cases}
x'=x\log(b)-y \\ y'=y\log(b)+x 
\end{cases}
\end{equation}

determinata con una ricerca in rete. Di seguito la function che ne disegna la soluzione, preso in inputo il fattore $b$ di decrescita della spirale:

\begin{lstlisting}
function []=spiralecartesio(b)
f=@(x,y) [y(1)*log(b)-y(2);y(2)*log(b)+y(1)];
sol1=@(teta) b.^teta.*cos(teta);
sol2=@(teta) b.^teta.*sin(teta);
[t,x]=ode45(f,[0,150],[1;0]);
y1=sol1(t);
y2=sol2(t);
plot(x(:,1),x(:,2),y1,y2,'--');
axis equal
axis([-1 1 -1 1])
title('Spirale di Cartesio')
end
\end{lstlisting}

La figura della spirale, disegnata mediante la soluzione del sistema, è la seguente:

\begin{figure}[!h]
\centering
\includegraphics[width=16cm]{figura_7_2.jpg}
\end{figure}



\end{document}