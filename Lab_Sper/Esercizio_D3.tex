\documentclass{article}
\usepackage[a4paper, top=3cm, left=3cm, bottom=3cm, right=3cm]{geometry}
\usepackage[T1]{fontenc} 
\usepackage[utf8]{inputenc} 
\usepackage[italian]{babel} 
\usepackage{lipsum} 
\usepackage{url} 
\usepackage{lmodern}
\usepackage{graphicx}
\usepackage{psfrag}
\usepackage{amsfonts}
\usepackage{amsthm}
\usepackage{amsmath}
\usepackage{amssymb}
\usepackage{mathrsfs}
\usepackage{tikz-cd}
\usepackage{mathtools}
\usepackage{tikz} 
\usepackage{caption}
\captionsetup{labelformat=empty,textfont=sl}
\usetikzlibrary{decorations.markings}
\usepackage{enumitem}
\usepackage[hidelinks]{hyperref}
\usepackage[numbered,framed]{matlab-prettifier}

\lstset{
  style              = Matlab-editor,
  basicstyle         = \mlttfamily,
  escapechar         = ",
  mlshowsectionrules = true,
}

\title{\textbf{Laboratorio Sperimentale di Matematica Computazionale / / Trasformazione Cilindrica - D3}}
\author{Dario Rancati - 539365} 


\begin{document}

\maketitle

\noindent
La trasformazione da implementare ha il primo, evidente, problema che l'immagine della trasformazione considerata (che dipende da una cotangente) ha potenzialmente coordinata x infinita se trasformiamo l'intera immagine di partenza. Per ovviare a questo problema abbiamo arbitrariamente scelto che la dimensione orizzontale dell'immagine trasformata sia pari a dieci volte quella dell'immagine in partenza. Abbiamo sostituito l'arcocotangente \texttt{acot} di matlab con un arcocotangente traslato \texttt{acot1} che trasla la soluzione di $\pi$ ogni volta che questa è negativa, perchè avevamo calcolato così la formula e sembrava più semplice di cambiarla usando il codominio di \texttt{acot}.

\begin{lstlisting}
function Y=proiettaD3(X)
[n,m]=size(X);
%qua innanzitutto capiamo la larghezza della proiezione
%e  ne settiamo le dimensioni
h=floor(cot(pi/m)); 
Y=zeros(n,10*m);
%qua mettiamo la formula trigonometrica nella coordinata j
%ovviamente ip e jp sono le coordinate ausiliarie per scrivere quelle di Y
for i=1:n
    for j=1:10*m
        ip=i;
        jp=floor(m*(1-acot1(h*(2*j/(10*m)-1))/pi));
        % ora controlliamo che le coordinate siano nell'immagine di partenza
        if jp>0 && jp<=m
        Y(i,j)=X(ip,jp);
        end
    end
end
imagesc(Y);
end
\end{lstlisting}

\noindent
Mettiamo di seguito l'immagine di un insieme di Mandelbrot e della sua immagine mediante la funzione

\begin{figure}[h!]
\centering
\includegraphics[width=\textwidth]{Mandelbrot.jpg}
\end{figure}


\begin{figure}[h!]
\centering
\includegraphics[width=\textwidth]{Mandelbrot_G2.jpg}
\end{figure}

\end{document}