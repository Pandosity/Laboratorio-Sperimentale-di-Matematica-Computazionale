\documentclass{article}
\usepackage[a4paper, top=3cm, left=3cm, bottom=3cm, right=3cm]{geometry}
\usepackage[T1]{fontenc} 
\usepackage[utf8]{inputenc} 
\usepackage[italian]{babel} 
\usepackage{lipsum} 
\usepackage{url} 
\usepackage{lmodern}
\usepackage{graphicx}
\usepackage{psfrag}
\usepackage{amsfonts}
\usepackage{amsthm}
\usepackage{amsmath}
\usepackage{amssymb}
\usepackage{mathrsfs}
\usepackage{tikz-cd}
\usepackage{mathtools}
\usepackage{tikz} 
\usepackage{caption}
\captionsetup{labelformat=empty,textfont=sl}
\usetikzlibrary{decorations.markings}
\usepackage{enumitem}
\usepackage[hidelinks]{hyperref}
\usepackage[numbered,framed]{matlab-prettifier}

\lstset{
  style              = Matlab-editor,
  basicstyle         = \mlttfamily,
  escapechar         = ",
  mlshowsectionrules = true,
}

\title{\textbf{Laboratorio Sperimentale di Matematica Computazionale / / Esercitazione 4}}
\author{Dario Rancati} 


\begin{document}

\section*{Esercizio 1}

Scriviamo uno script che risolva con \texttt{ode45} e disegni il grafico dell'equazione del testo in quattro casi diversi a seconda dei parametri utilizzati.
Utilizziamo ovviamente i parametri del testo anche per i parametri dell'equazione e per l'intervallo di integrazione.

\begin{lstlisting}
f=@(t,n) [0.21827*(1-n(1)/13-n(2)/3.71429)*n(1); 0.06069*(1-n(2)/5.8-n(1)/13.2118)*n(2)];
[t1,sol1]=ode45(f,[0 300], [0.5; 0.3]);
[t2,sol2]=ode45(f,[0 300], [0.5; 0.5]);
[t3,sol3]=ode45(f,[0 300], [0.1; 0.5]);
[t4,sol4]=ode45(f,[0 300], [10; 0.5]);
subplot(2,2,1)
plot(t1,sol1(:,1),t1,sol1(:,2))
legend('N1','N2')
title('N1(0)=0.5, N2(0)=0.3')
subplot(2,2,2)
plot(t2,sol2(:,1),t2,sol2(:,2))
legend('N1','N2')
title('N1(0)=0.5, N2(0)=0.5')
subplot(2,2,3)
plot(t3,sol3(:,1),t3,sol3(:,2))
legend('N1','N2')
title('N1(0)=0.1, N2(0)=0.5')
subplot(2,2,4)
plot(t4,sol4(:,1),t4,sol4(:,2))
legend('N1','N2')
title('N1(0)=10, N2(0)=0.5')
\end{lstlisting}

\begin{figure}[!h]
\centering
\includegraphics[width=\textwidth]{figura_4_1.jpg}
\end{figure}

Notiamo che $N2$ tende a $5.8,$ e che $N1$ tende a zero in tutti e quattro i casi.

\newpage

\section*{Esercizio 2}


La seguente function implementa il modello di Lotka-Volterra in funzione di due parametri $t_{max}$ e del passo $h$ del metodo di Runge-Kutta usato nella risoluzione dello stesso.

\begin{lstlisting}
function []=lotkavolterra(tmax,h)
alfa=2;
gamma=0.001;
alfa1=1;
beta1=0.001;
fun=@(t,x) [x(1)*(alfa-gamma*x(2)); x(2)*(-alfa1+beta1*x(1))];
[t1,u1]=RK4(fun,[0:tmax],[300;150],h);
[t2,u2]=RK4(fun,[0:tmax],[15;22],h);

subplot(2,2,1)
plot(t1,u1(1,:),'r',t1,u1(2,:),'b');
legend('Preda','Predatore');
title('Inizio:300,150');

subplot(2,2,2)
plot(t2,u2(1,:),'r',t2,u2(2,:),'b');
legend('Preda','Predatore');
title('Inizio:15,22');

subplot(2,2,3)
plot(u1(1,:),u1(2,:))
xlabel('Preda')
ylabel('Predatore')
title('Inizio:300,150');

subplot(2,2,4)
plot(u2(1,:),u2(2,:))
xlabel('Preda')
ylabel('Predatore')
title('Inizio:15,22');
\end{lstlisting}

\noindent
Abbiamo settato $t_{max}=50$ espresso in ore e, come ormai standard, $h=0.01$. Osserviamo il simpatico fenomeno per cui i predatori consumano interamente la loro scorta di cibo per poi estinguersi, far ri-aumentare il numero di prede e riconsumarli tutti nuovamente. Questo comportamento è evidente anche dai due grafici inferiori che descrivono le prede in funzione dei predatori, che sono chiaramente ciclici.

\begin{figure}[!h]
\centering
\includegraphics[width=\textwidth]{figura_4_2.jpg}
\end{figure}

\clearpage

\section*{Esercizio 3}

Esattamente come nella funzione precedente, anche qui implementiamo il modello di Lotka-Volterra, ma stavolta lo risolviamo con \texttt{ode45}. Abbiamo usato il comando \texttt{find} per risolvere un solo problema differenziale e limitarci a ridurre l'intervallo di integrazione a manina, invece di far partire un'altra routine di \texttt{ode45}.
\begin{lstlisting}
f=@(t,x) [0.405*x(1)-0.81*x(1)*x(2); -1.5*x(2)+0.125*x(1)*x(2)];
[t,sol]=ode45(f,[0 25], [7.7; 0.5]);
T=find(t>10,1);
subplot(1,2,1) 
plot(sol(1:T,1),sol(1:T,2));
xlabel('Gnu')
ylabel('Leoni')
subplot(1,2,2)
plot(t,sol(:,1),t,sol(:,2));
legend('Gnu','Leoni')
\end{lstlisting}

\noindent
Le considerazioni qualitative sono molto simili a quelle dell'esercizio precedente, non fosse che la popolazione delle prede non si azzera mai, ma tende nel momento di massima proliferazione della popolazione di predatori al valore $8$. Per il resto, il comportamento è ancora oscillatorio.

\begin{figure}[!h]
\centering
\includegraphics[width=\textwidth]{figura_4_3.jpg}
\end{figure}

\newpage

\subsection*{Esercizio 4}
L'equazione differenziale che descrive il fenomeno \`{e} $x(t)'=rx(t)y(t),y(t)'=-rx(t)y(t),$ dove $x(t)$ sono gli infetti e $y(t)$ sono i malati. La malattia non e' mortale, ma tende a cronicizzarsi: detto r il tasso di infezione, la function seguente restituisce il tempo necessario ad avere il numero di infetti pari ai 4/5 della popolazione in funzione di $N$ numero di individui e $i_{0}$ numero di infetti iniziali.
\begin{lstlisting}
function y=malattia(i0,r,N,T)
f=@(t,x) [r*x(1)*x(2);-r*x(1)*x(2)];
[t,sol]=ode45(f,[0 T], [i0; N-i0]);
plot(t,sol(:,1),'r',t,sol(:,2),'g');
legend('Infetti','Malati');
l=find(sol(:,1)>0.8*N,1);
y=t(l);
end
\end{lstlisting}
Abbiamo posto, come da testo, $i_{0}=1$, $N=1000$ e $r=0.0001$. Inoltre abbiamo integrato con $T=200$. Il risultato ottenuto è $83.3536$ ore, e il grafico è il seguente:
\begin{figure}[!h]
\centering
\includegraphics[width=\textwidth]{figura_4_4.jpg}
\end{figure}



\end{document}