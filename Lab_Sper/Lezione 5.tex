\documentclass{article}
\usepackage[a4paper, top=3cm, left=3cm, bottom=3cm, right=3cm]{geometry}
\usepackage[T1]{fontenc} 
\usepackage[utf8]{inputenc} 
\usepackage[italian]{babel} 
\usepackage{lipsum} 
\usepackage{url} 
\usepackage{lmodern}
\usepackage{graphicx}
\usepackage{psfrag}
\usepackage{amsfonts}
\usepackage{amsthm}
\usepackage{amsmath}
\usepackage{amssymb}
\usepackage{mathrsfs}
\usepackage{tikz-cd}
\usepackage{mathtools}
\usepackage{tikz} 
\usepackage{caption}
\captionsetup{labelformat=empty,textfont=sl}
\usetikzlibrary{decorations.markings}
\usepackage{enumitem}
\usepackage[hidelinks]{hyperref}
\usepackage[numbered,framed]{matlab-prettifier}

\lstset{
  style              = Matlab-editor,
  basicstyle         = \mlttfamily,
  escapechar         = ",
  mlshowsectionrules = true,
}

\title{\textbf{Laboratorio Sperimentale di Matematica Computazionale / / Esercitazione 5}}
\author{Dario Rancati} 


\begin{document}


\section*{Esercizio 1}

La seguente function restituisce il tempo che il paracadutista impega ad atterrare, risolvendo l'ovvia equazione del moto del paracadutista con il metodo di Runge-Kutta. Tale equazione è la seguente:
$$y''(x)=-\frac{k_{1/2}}{m}y'(x)-g$$
\noindent
con $k$ uguale al valore corrispondente alla fase in cui si trova il paracadutista (con o senza paracadute). Abbiamo cercato l'intervallo di integrazione opportuno per ottenere il punto di atterraggio, che si rivelerà essere poco sotto a 150,per cui abbiamo scelto 150 come estremo dell'intervallo di integrazione.
\begin{lstlisting}
function xmax=paracadutista(h)
k1=16.4;
k2=180;
g=9.81;
m=90;
xp=15;
alt=1200;

f1=@(x,y) [y(2); -(k1/m)*y(2)-g];
[x1,y1]=RK4(f1, [0, xp], [alt; 0], h);
n=length(x1);
f2=@(x,y) [y(2); -(k2/m)*y(2)-g];
[x2,y2]=RK4(f2, [xp, 150], [y1(1,n); y1(2,n)], h);

y=[y1,y2];
x=[x1,x2];
l=find(y(1,:)<0,1);
xmax=x(l);

subplot(2,1,1) 
plot(x(1:l),y(1,1:l))
axis([0 xmax 0 1200])
legend('Posizione')
xlabel('Tempo')

subplot(2,1,2) 
plot(x(1:l),y(2,1:l))
legend('Velocita')
xlabel('Tempo')
end
\end{lstlisting}

\noindent
La function riporta come tempo di atterraggio $x_{max}=146.700$. Facendo calcolare a matlab anche $v_{max}=y_{2}(l)$ la velocità al tempo di atterraggio otteniamo $-4.9050$, che rimane costante da quando è trascorso il transiente post apertura del paracadute in poi, come si nota dal grafico seguente.

\newpage

\begin{figure}[!h]
\centering
\includegraphics[width=\textwidth]{figura_5_1.jpg}
\end{figure}

\newpage

\section*{Esercizio 2}

L'equazione che dobbiamo risolvere mediante \texttt{ode45} è la seguente:
$$y''=\frac{g}{l}sin(y)$$
\noindent
e ne grafichiamo posizione e velocità

\begin{lstlisting}
y0=pi/6;
y1=0;
l=2;
f=@(t,y) [y(2);-9.81/l*sin(y(1))];
[t,sol]=ode45(f,[0 10],[y0;y1]);
plot(t,sol(:,1),t,sol(:,2));
legend('Posizione','Velocita');
xlabel('Tempo')
\end{lstlisting}

\begin{figure}[!h]
\centering
\includegraphics[width=\textwidth]{figura_5_2.jpg}
\end{figure}

\newpage

\subsection*{Esercizio 3}

L'equazione che dobbiamo risolvere mediante \texttt{ode45} è la seguente:
$$y''=\frac{g}{l}y$$
\noindent
e ne grafichiamo posizione e velocità

\begin{lstlisting}
y0=pi/6;
y1=0;
l=2;
f=@(t,y) [y(2);-9.81/l*y(1)];
[t,sol]=ode45(f,[0 10],[y0;y1]);
plot(t,sol(:,1),t,sol(:,2));
legend('Posizione','Velocita');
xlabel('Tempo')
\end{lstlisting}

\noindent
Non notiamo differenze significativa (almeno ad occhio nudo) tra questo modello e quello precedente.

\begin{figure}[!h]
\centering
\includegraphics[width=\textwidth]{figura_5_3.jpg}
\end{figure}

\newpage

\subsection*{Esercizio 4}

Per prima cosa scriviamo una function che, assegnati massa, costante elastica, costante di smorzamento, forza applicata, intervallo di tempo e valori iniziali plotti posizione e velocità dell'oscillatore armonico con i dati in questione:

\begin{lstlisting}
function []=oscillatore(m,h,k,F,intervallo,y0,y1)
f=@(t,y) [y(2);-h/m*y(1)-k/m*y(2)+F];
[t,sol]=ode45(f,intervallo,[y0;y1]);
plot(t,sol(:,1),t,sol(:,2));
legend('Posizione','Velocita');
xlabel('Tempo')
end
\end{lstlisting}

\noindent
Scriviamo uno script che consenta di scegliere il caso test con un ciclo \texttt{switch}. Abbiamo usato \texttt{menu} perchè sembrava di implementazione enormemente più compatta, anche se probabilmente la differenza è minima.

\begin{lstlisting}
choice = menu('Scegli un tipo di oscillatore:','Libero non smorzato','Libero sottosmorzato','Libero sovrasmorzato','Forzato smorzato');
switch choice
    case 1
        oscillatore(1,10,0,0,[0 60],1,0);
    case 2
        oscillatore(1,10,0.5,0,[0 60],1,0);
    case 3
        oscillatore(1,10,10,0,[0 60],1,0);
    case 4
        oscillatore(2,15,0.75,25,[0 60],2,0);
end
\end{lstlisting}

\begin{figure}[!h]
\centering
\includegraphics[width=\textwidth]{figura_5_4_1.jpg}
\caption{Libero non smorzato}
\end{figure}

\begin{figure}[!h]
\centering
\includegraphics[width=\textwidth]{figura_5_4_2.jpg}
\caption{Libero sottosmorzato}
\end{figure}

\begin{figure}[!h]
\centering
\includegraphics[width=\textwidth]{figura_5_4_3.jpg}
\caption{Libero sovrasmorzato}
\end{figure}

\begin{figure}[!h]
\centering
\includegraphics[width=\textwidth]{figura_5_4_4.jpg}
\caption{Forzato smorzato}
\end{figure}

\clearpage


\end{document}